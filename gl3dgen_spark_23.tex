% Intended LaTeX compiler: pdflatex
\documentclass[a4paper,10pt]{article}
\usepackage[utf8]{inputenc}
\usepackage[T1]{fontenc}
\usepackage{graphicx}
\usepackage{longtable}
\usepackage{wrapfig}
\usepackage{rotating}
\usepackage[normalem]{ulem}
\usepackage{amsmath}
\usepackage{amssymb}
\usepackage{capt-of}
\usepackage{hyperref}
\usepackage[margin=1.2in]{geometry}
\newgeometry{top=1in,bottom=1in,right=1.25in,left=1.25in}
\usepackage{setspace}
\onehalfspacing
\usepackage{titling}
\setlength{\droptitle}{-0.75in}
\usepackage{natbib}
\usepackage{makeidx}
\usepackage{hyperref}
\usepackage{cleveref}
\usepackage[dvipsnames]{xcolor}
\usepackage{parskip}
\usepackage{bm}
\date{}
\title{Deep learning the generation of gravitational lensing deflection fields from 3D models of galaxies}
\hypersetup{
 pdfauthor={phdenzel},
 pdftitle={Deep learning the generation of gravitational lensing deflection fields from 3D models of galaxies},
 pdfkeywords={},
 pdfsubject={},
 pdfcreator={Emacs 30.0.50 (Org mode 9.6.1)}, 
 pdflang={English}}
\begin{document}

\maketitle
\vspace{-0.8in}
\section*{Project summary}
\label{sec:org114fcd3}

Gravitational lensing is a phenomenon that occurs when rays of light
from a distant background source are deflected by the gravitational
field of a massive foreground object, e.g. a galaxy, which almost
perfectly aligns with the observer. While such occurrences are rare,
they are scientifically significant, because they provide the only
opportunity to directly infer the lensing galaxy's mass distribution,
including its dark matter content.  This unique perspective on a
galaxy's dark matter distribution offers exceptional insights into the
mysteries surrounding galaxy evolution, the nature of dark matter,
galaxy substructures, and even the expansion of the Universe. \\[0pt]
Accurately predicting the deflection field of a strong gravitational
lens is a complex task that requires a detailed understanding of the
distribution of matter in the lensing galaxy. Conventional methods for
calculating these deflection fields, such as ray-tracing, are
computationally expensive, can take a long time to generate results,
and typically have to be fine-tuned by an experienced expert.

In recent years, deep learning methods have emerged as a promising
approach for generating and processing image-based data in various
scientific fields, often with super-human proficiency. These methods
employ neural networks to learn the mapping between the input
properties (for instance, the lensing galaxy) and the resulting image
(in this instance, the deflection field).

In this research proposal, I outline the usage of generative deep
learning methods to produce strongly lensing deflection fields of 3D
galaxy models from existing hydrodynamical simulation suites for the
purpose of creating mock observations, observational fits, and
corresponding source reconstructions. Specifically, I will explore
various state-of-the-art deep learning architectures, including
diffusion models, flow-based models, generative adversarial networks
(GANs), and variational autoencoders (VAEs) to develop a model that
can accurately generate the deflection field from a given 3D galaxy
model. The resulting deep learning model will be used to create
synthetic observations that can be compared to existing observational
data to test the accuracy of the lens model, and in particular
investigate the theoretical properties of the observed lensing
galaxies and their corresponding background source reconstructions,
within a Bayesian framework.

Lens modelling is inherently considered as a (degenerate) 2D inverse
problem. The novelty of this project consists of introducing 3D models
as direct input, which requires methods able to cover a broad range in
solution space due to the additional degeneracies introduced
thereby. Conventional modelling methods are insufficient due to the
typically low complexity of their models whereas deep learning methods
increase the model complexity with a high number of parameters, thus
able to span a wider range in solution space.

As of today, roughly 10\textsuperscript{3} lenses have been discovered, only a fraction
of those properly analysed. Moreover, it is anticipated that the
next-generation satellites and telescopes such as
JWST\footnote{James~Webb~Space~Telescope}, Euclid,
SKA\footnote{Square~Kilometer~Array}, or
ELT\footnote{Extremely~Large~Telescope} may increase this number to
10\textsuperscript{5}. Thus, it is crucial to devise novel techniques that can scale
with big data efficiently.


\newpage
\section*{Project plan}
\label{sec:orgb439678}

\subsection*{State of research}
\label{sec:org7dd05ca}

Although gravitational lensing is often thought of as an illusion,
showing the same source in multiple images, lensing systems have
distinct visual features which make them relatively easy to spot by
the trained eye. However, wide-field surveys record billions of light
sources, from which over 100'000 strongly lensing systems may still
await discovery \citep{Taak20,Taak23,Collett15}, and not all
instances can be checked with same amount of attention by humans.

Deep learning techniques have emerged as a promising approach for the
automatic discovery of strong gravitational lenses. In fact, the
number of detections and identifications of strongly lensing
candidates has more than tripled through the usage of deep learning
techniques \citep{Storfer22,Huang21,Rezaei22,Wilde22}.

For lens modelling, the situation is different. \cite{Young80}
simultaneously confirmed and modelled the very first lens discovery by
\cite{Walsh79}. They used a simple spherically symmetric lens model
using 4 parameters to fit the observation. Ever since then, even
modern lens modelling tools employ some type of parametric model (with
typically no more than 8-12 parameters) to fit the density
distribution of the lens
\citep[cf.][]{Birrer18,Hezaveh17,Tessore16,Oguri10}. Note that
\cite{Hezaveh17} use deep learning convolutional neural networks
(CNNs) to fit mass distribution parameters to lens
observations. However, this still prescribes the same limits on the
complexity of the lensing galaxy's shape.

The often ignored issue with parametric models is the assumption that
these models are able to cover enough 

\begin{itemize}
\item Issue of degeneracy

\item free-form lens modelling use over parametrization

\item 
\end{itemize}

Another type of lens modelling technique, the so-called
free-form lens 




\begin{itemize}
\item first modelling by Young et al, parametric, degeneracies, too simple
\item free-form models not much better in producing realistic models
\end{itemize}

\begin{itemize}
\item in comes deep learning\ldots{}
\item a few deep learning architectures
\item 
\end{itemize}


\begin{itemize}
\item novelty distinguishing from the rest
\end{itemize}

\subsection*{Project description}
\label{sec:orgbf86ea5}

\begin{itemize}
\item Goals

\item Methods
\end{itemize}

\begin{equation}
\label{eq:thinlens}
  \kappa(\bm\theta,\xi) = \frac{4\pi G}{cH_0}\, \frac{D_\mathrm{LS}D_\mathrm{L}}{D_\mathrm{S}} \int \rho(\bm\theta,\xi,z)\,\mathrm{d}z \,.
\end{equation}

\begin{equation}
  \alpha(\bm\theta,\xi) = \nabla_{\bm\theta}\psi(\bm\theta,\xi) = 2\nabla^{-1}\kappa(\bm\theta,\xi)
\end{equation}


\begin{itemize}
\item Approach
\item Expected results
\item Possible risks
\end{itemize}


\subsection*{Potential impact}
\label{sec:org9a00517}



\bibliographystyle{apsrev}
\bibliography{gl3dgen}
\end{document}